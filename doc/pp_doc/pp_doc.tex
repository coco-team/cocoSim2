\documentclass[a4paper,12pt]{article}

\usepackage[utf8]{inputenc}
\usepackage[francais]{babel}
\usepackage[T1]{fontenc}
\usepackage{geometry}
%Defining the header
\usepackage{fancyhdr}
\pagestyle{fancy}
\fancyhead[L]{Pre-GAL Documentation}
\fancyhead[R]{Autumn 2014}

\geometry{
 body={15.5cm, 21.6cm},
 left=3cm,
 top=3cm,
 bottom=3cm
}

\begin{document}
\title{Pre-GAL Documentation}
\author{Romain LECONTE}
\date{v1.1}
\maketitle
\newpage

\section{User Manual}

\subsection{Requirements}
\begin{itemize}
\item
Matlab r2014b or newer
\item
Simulink
\item
Python
\end{itemize}

\subsection{Instructions}
To use Pre-GAL which converts a Simulink model into a GAL-friendly one, open Matlab and add the folder where the script \texttt{PreGAL.m} is to the Matlab path (usually add \texttt{'preprocessing\_scripts'} using the right click and the option \texttt{'Add to path'} then \texttt{'Selected Folders and subfolders'}). \newline
Then go to the folder where is located the Simulink model you want to process (using \texttt{cd} or the Matlab browser) and type the following command :
$$\texttt{PreGAL('filename')}$$

If the Simulink model needs the definition of constants in order to run, you have to give a \texttt{.m} file containing these values to PreGAL  with the command :
$$\texttt{PreGAL('filename','constant\_filename')}$$

You have to give the complete name of the files, including the extension as follow :
$$\texttt{PreGAL('model.slx')} \hspace{0.1cm} or \hspace{0.1cm}
\texttt{PreGAL('model.slx','model\_constants.m')}$$

PreGAL will create :
\begin{itemize}
\item
\texttt{modelname\_p.slx} or \texttt{.mdl} file which contains the GAL-friendly model in the same file format as the model processed.
\item
\texttt{modelname\_GAL.mdl} file which contains the GAL-friendly model in the Matlab r2008b compatible file format, which is supported by GAL.
\end{itemize}

\subsubsection*{Options}
You can call Pre-GAL with the \texttt{'verif'} option which will create some simulink models containing every block processed and the block generated by Pre-GAL side by side. 
Those models are stored in the "Verification" folder. The command is :
$$\texttt{PreGAL('model.slx','model\_constants.m','verif')}$$
You can also use $\texttt{PreGAL('model.slx','','verif')}$ if no constant file is required.

\subsection{Details}
PreGAL needs to compile the model during its execution, if it lacks values of certain parameters for example, be aware that the preprocessing will fail. \newline

Matlab r2014b is required because of its python capabilities.
If you want to use Pre-GAL in an older version of Matlab, you can, but any call of the Python parser script will fail and you will get error messages in the console each time this python script is called.
That means that blocks needing parsing to be processed (fcn, gain and constants with calculation for example) won't be processed, nevertheless all other blocks will be processed normally.


\section{Tree structure of source folders}

The whole program is designed with Matlab functions. The main Matlab function is PreGAL written in the \texttt{PreGAL.m} file.
All the functions of the program that you can find in the \texttt{lib} folder are called by the master function \texttt{PreGAL}. \newline

In the folder \texttt{blocks} you will find one function for each kind of block that is handled by Pre-GAL. \newline
In the folder \texttt{common} you will find the functions used to handle several blocks. A good example is the discrete state space process, which is used for transfer function blocks, zero-pole blocks and of course discrete state space blocks. \newline
In the folder \texttt{math} you will find functions needed to handle any math expression. In this folder you also have a Python script which is called from Matlab, and that's why the r2014b or newer release is required to run Pre-GAL properly.

\section{Supported blocks}

The blocks registered in this section are converted into a GAL-friendly equivalent, but certain option may not be handled by the translation.
Some of the blocks are complex and need a very customized handling, the blocks to replace them are generated by matlab scripts.
Other blocks are very similar each time you find them in a model, these blocks are replaced by generic blocks that are contained in the file \texttt{gal\_lib.slx} contained in the \texttt{common} folder.
~\\

\begin{itemize}
\item
\textbf{Goto/From} : Create lines between blocks that have the same "tag". \newline
	Supported options :
	\begin{itemize}
	\item
	\texttt{TagVisibility}: global/local
	\end{itemize}
Details : This function also create inputs/outputs to some subsystems in order to link together the blocks. 
~\\
\item
\textbf{FromWorkspace} : Replaced by an input in the first level system. \newline
The goal is to provide this input as a variable for PKind tests.
~\\
\item
\textbf{ToWorkspace} : Replaced by an output in the first level system.
~\\
\item
\textbf{Function} : Replaced by a subsystem containing the calculation made by blocks. \newline
Details : a python parser called from Matlab gives a tree of the mathematical expression which is converted into blocks with only variables and figures stored in constant blocks. \newline
The the parser recognize any function by default (of the form 'func(arg)'), the handling of new functions has to be added in \texttt{expression\_process} located in \texttt{Processes/math}. \newline
\item
\textbf{Math} : Replace Math blocks with yet unsupported operations. \newline
	Operations handled by the script (not handled natively by GAL):
	\begin{itemize}
	\item
	\texttt{magnitude\char`\^2}: replaced by a \texttt{pow} using the parsing operation on 'u\char`\^ 2'
	\end{itemize}
	~
\item
\textbf{Selector} : Replace by a subsystem block containing a demux of the input, a mux of the output and link to the required inputs. \newline
	Supported options:
	\begin{itemize}
	\item
	\texttt{IndexMode}: Zero-based/One-based
	\item
	\texttt{IndexOptionArray}: 'Index vercor (dialog)' only supported. \newline
	new options may easily be added.
	\end{itemize}
	~
\item
\textbf{TransferFunction} : Replace by a GAL-friendly block generated from the state space representation provided by Matlab from the numerator and denominator of the function.
~\\
\item
\textbf{Zero-Pole} : Replace by a GAL-friendly block generated from the state space representation provided by Matlab from the poles and zeros of the function.
~\\
\item
\textbf{DiscreteStateSpace} : Replace by a GAL-friendly block generated from the state space representation.
~\\
\item
\textbf{Lookup tables} : Replaced by a subsystem providing an interpolation based on table provided (from the initial block) during the process.
	Supported options:
	\begin{itemize}
	\item
	\texttt{InputValues}
	\item
	\texttt{Table}
	\end{itemize}
	~
\item
\textbf{Lookup tables nD} : Replaced by a subsystem providing an interpolation based on table provided (from the initial block) during the process.
	Supported options:
	\begin{itemize}
	\item
	\texttt{NamberOfTableDimensions} : 1
	\item
	\texttt{BreabpointsForDimension1}
	\item
	\texttt{Table}
	\end{itemize}
Details : One of the possible enhancement would be to handle tables with more than 1 dimension.
~\\
\item
\textbf{Clock} : Replace by a GAL-friendly block provided in the \texttt{gal\_lib.slx} located in \texttt{Processes/common}. \newline
Details : The clock is basically an integrator of the constant 1.
~\\
\item
\textbf{Integrator} : Replace by a GAL-friendly block provided in the \texttt{gal\_lib.slx} located in \texttt{Processes/common}. \newline
	Supported options :
	\begin{itemize}
	\item
	\texttt{InitialConditionSource}: internal/external
	\end{itemize}
	~
\item
\textbf{Discrete Integrator} : Replace by a GAL-friendly block provided in the \texttt{gal\_lib.slx} located in \texttt{Processes/common}. \newline
	Supported options :
	\begin{itemize}
	\item
	\texttt{InitialConditionSource}: internal/external
	\end{itemize}
Details : The blocks provided in \texttt{gal\_lib.slx} have the expected behavior if the sample time of sum blocks are respected by GAL, but GAL does not support these parameters yet, subsequently the behavior of the block may be a little different than the original.
~\\
\item
\textbf{Dead Zone} : Replace by a GAL-friendly block provided in the \texttt{gal\_lib.slx} located in \texttt{Processes/common}. \newline
	Supported options: 
	\begin{itemize}
	\item
	\texttt{LowerValue}
	\item
	\texttt{UpperValue}
	\end{itemize}
	~\\
\textbf{Dynamic Dead Zone} : Replace by a GAL-friendly block provided in the \texttt{gal\_lib.slx} located in \texttt{Processes/common}. \newline
	Supported options: 
	\begin{itemize}
	\item
	\texttt{LowerValue}
	\item
	\texttt{UpperValue}
	\end{itemize}
Details : To modify the behavior of these blocks, you must modify their replacements in \texttt{gal\_lib.slx}. 
~\\
\item
\textbf{Saturation} : Replace by a GAL-friendly block provided in the \texttt{gal\_lib.slx} located in \texttt{Processes/common}. \newline
	Supported options: 
	\begin{itemize}
	\item
	\texttt{LowerLimit}
	\item
	\texttt{UpperLimit}
	\end{itemize}
	~\\
\textbf{Dynamic Saturation} : Replace by a GAL-friendly block provided in the \texttt{gal\_lib.slx} located in \texttt{Processes/common}. \newline
Supported options: 
	\begin{itemize}
	\item
	\texttt{LowerLimit}
	\item
	\texttt{UpperLimit}
	\end{itemize}
Details : To modify the behavior of these blocks, you must modify their replacements in \texttt{gal\_lib.slx}. 
~\\
\item
\textbf{RateTransition} : Replaced by a GAL-friendly block provided in the \texttt{gal\_lib.slx} located in \texttt{Processes/common}. \newline
	Supported options:
	\begin{itemize}
	\item
	\texttt{OutPortSampleTime}
	\item
	\texttt{X0}
	\end{itemize}
Details : The block allow to output the \texttt{X0} initial condition during the first step of execution of the Lustre code. The output is also updated at a \texttt{OutPortSampleTime} rate, this behavior will be effective when GAL will consider sample times provided in each block.
If GAL don't consider different sample times, this block has no purpose other than the initial condition.
\end{itemize}

\section{Comments}

Matlab can add some comments in mdl files for its own compiler "RealTimeWorkshop" known as "Simulink Coder" in the latest releases. These comments are not supported by GAL and have to be removed before processing the model.

In order to do that, you can enable the required option in the simulink preferences IF you have a licence for Simulink Coder.

In order to avoid the need of this licence, those comments are removed manually in the mdl file (this operation on the mdl file is performed at the end of the main function). \newline

GAL needs to type every signal in the Simulink model. Usually, automatic control engineer don't type anything in Simulink models and keep the option in \texttt{inherit:auto}. GAL has a typer than can propagate the type of input threw the system, but needs the inputs to be typed. That's why by default, all import blocks that are not typed in the simulink model are typed as \texttt{double} (this operation is performed at the end of the main function). \newline

The blocks \texttt{FromWorkspace} are replaced by an import in the top level layer of the model, and blocks \texttt{ToWorkspace} are replaced by outports in the top level layer of the model. But these blocks can sometimes be used to load some precise data into the model (curves from experiments for example), and in this case, the model generated won't support this input of information. The block should be replaced by a lookup table with a clock in input for example if we want this data to go threw the compilation. \newline

Pre-GAL calls a Python script to parse mathematical expression. The capability of calling Python scripts has been enabled in the r2014b release (September 2014), that's why Pre-GAL requires this Matlab version to be completely operational. Pre-GAL can be used without this version, but the function blocks and gain/constant blocks with complex calculation won't be processed.

\end{document}






















